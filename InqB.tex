\section{Inquisitive Semantics}\label{sec: InqB}
% Language
Below we will give a brief overview of \textsf{InqB}, taking our cue from \cite{inquisitive19}.\footnote{Basic notions such as propositions, information states and alternatives are treated comprehensively in Chapter 2 of \cite{inquisitive19}, while the language, models and semantics of \textsf{InqB} are treated in Chapter 4 of the same book.}

Any standard first-order language $\mathcal{L}$, containing a set of function symbols $\mathcal{F}_\mathcal{L}$ and a set of relation symbols $\mathcal{R}_\mathcal{L}$, is also a language of \textsf{InqB}. In our model checker we do not concern ourselves with function symbols, therefore we will not mention them in the remainder of this report. As for the constants in a language, we will assume that for each individual in the domain of a model we will have a constant in our language. We define models of \textsf{InqB} below.

% Model
\begin{defi}\label{def: InqBModel}
An \textsf{InqB} model for a first-order language $\mathcal{L}$ is a triple $M=\langle W,D,I\rangle$, where:
\begin{itemize}
\setlength\itemsep{-0.3em}
    \item $W$ is a non-empty set of possible worlds;
    \item $D$ is a non-empty set of individuals;
    \item $I$ is a map that associates every $w\in W$ with a first order structure $I_w$ such that:
    \begin{itemize}
    \setlength\itemsep{-0.3em}
        \item for every $w\in W$, the domain of $I_w$ is $D$;
        \item for every $n$-ary relation symbol $R\in \mathcal{R}_{\mathcal{L}}$, $I_w(R)\subseteq D^n$;
    \end{itemize}
\end{itemize}
\end{defi}

% info states
% propositions
Before giving the semantics of \textsf{InqB}, we introduce some terminology. Instead of worlds, inquisitive semantics takes sets of worlds as primitive. A set of worlds is called an information state. A proposition, then, consists of a set of sets of worlds instead of a set of worlds as in classical logic.

\begin{defi}\label{defprop}
 Let $M=\langle W,D,I\rangle$ be a model. An information state $s$ is a set of possible worlds $s\subseteq W$. A proposition $P$ is a non-empty, downwards closed set of information states. The proposition $P$ corresponding to a formula $\varphi$ (in a certain model) is denoted by $[\varphi]$.
\end{defi}

Given Definition \ref{defprop}, we can characterize a proposition in terms of its maximal elements, which we will call alternatives. 

\begin{defi}\label{defalt}
 The alternatives of a proposition $P$, denoted by $\alt(P)$, are its maximal elements. If $|\alt(P)|\neq 1$, we say that $P$ is inquisitive, otherwise $P$ is non-inquisitive.
\end{defi}

The issue raised by a proposition, then, can intuitively be understood as the problem of not knowing which of its alternatives is the case. The information states in a proposition correspond to the states in which the issue raised by a proposition is resolved, i.e. the states in which one is able to know for at least one alternative that it is the case. As smaller information states provide \emph{more} information, the requirement that propositions are downwards closed make sense. A proposition $P$ can be retrieved from its alternatives $\alt(P)$ by taking the downwards closure of $\alt(P)$.

If a proposition $P$ has more than one alternative and $P$ is inquisitive, it expresses the question which of its alternatives is the case. If there is only one alternative, then the proposition simply expresses the information contained in that single alternative, hence it makes sense that we call such a proposition non-inquisitive. Note that a proposition cannot have $0$ alternatives by definition.

Besides being inquisitive or not, propositions can be informative or non-informative. The union of the sets in a proposition can be viewed as the informative content of a proposition. If the informative content of a proposition does not contain all worlds, then we say that the proposition is informative: it provides us with the information that the excluded worlds are not the case. 

\begin{defi}\label{definfo}
 The informative content of a proposition $P$, denoted by $\info(P)$, is defined as $\info(P):=\cup P$. A proposition $P$ is informative if $\info(P)\neq W$ (where $W$ is the set of worlds in a model).
\end{defi}

% Algebraic foundations
In classical logic, logical operations correspond to certain algebraic operations. This is also the case in inquisitive semantics. We will use the algebraic characterization of \textsf{InqB}'s semantics in the implementation of our model checker, hence we also present the semantics in this way here.\footnote{For the semantics of \textsf{InqB} in terms of support conditions, see \cite[p.\ 62-63]{inquisitive19}.} In addition to union and intersection, two more algebraic operations are used: relative and absolute pseudo-complement.

\begin{defi}\label{defpseudo}
 For propositions $P$ and $Q$, the pseudo-complement of $P$ relative to $Q$, denoted by $P\Rightarrow Q$, is defined as $P\Rightarrow Q:= \{s \mid \text{for every } r\subseteq s\text{, if } t\in P\text{, then } t\in Q\}$. For any proposition $P$, the absolute pseudo-complement $P^*$ of $P$ is defined as $P^*:=\{s\mid s\cap t=\emptyset \text{ for all } t\in P\}$.
\end{defi}
% Semantics
Using these algebraic operations, we can now give the semantics of \textsf{InqB}.

\begin{defi}\label{defsemantics}
 The semantics of \textsf{InqB} are given by:
 \begin{enumerate}\setlength\itemsep{-0.3em}
     \item $[R(t_1,\dots,t_n)] := \mathcal{P}(|R(t_1,\dots,t_n)|)$;
     \item $[\neg \varphi]:=[\varphi]^*$;
     \item $[\varphi\land\psi]:=[\varphi]\cap [\psi]$;
     \item $[\varphi\lor\psi]:=[\varphi]\cup [\psi]$;
     \item $[\varphi\rightarrow\psi]:=[\varphi]\Rightarrow [\psi]$;
     \item $[\forall x . \varphi(x)]:= \cap_{d\in D} [\varphi(d)]$;
     \item $[\exists x . \varphi(x)]:= \cup_{d\in D} [\varphi(d)]$,
 \end{enumerate}
 where $|R(t_1,\dots,t_n)|$ denotes the set of worlds where $R(t_1,\dots,t_n)$ is classically true.
\end{defi}

The definition above gives us a semantics in the sense that every formula $\varphi$ is associated with a proposition $[\varphi]$, and that we can say that $\varphi$ is supported by an information state $s$ whenever $s\in [\varphi]$. If one prefers to speak about satisfaction in worlds, we can say that $\varphi$ is satisfied at a world $w$ if $w\in \info(\varphi)$.

% projection operators
Lastly, we introduce two new projection operators: `$!$' and `$?$'. We call these projection operators because they project a proposition on the set of purely non-inquisitive propositions and the set of purely non-informative propositions, respectively.

\begin{defi}\label{defproject}
 For any proposition $P$, we have:
 \begin{itemize}
     \item $!P:= \mathcal{P}(\info(P))$;
     \item $?P:= P\cup P^*$.
 \end{itemize}
\end{defi}

We can express $!P$ in terms of our algebraic operators as $!P=P^{**}$. It then follows from Definition \ref{defsemantics} and Definition \ref{defproject} that we can express the projection operators in terms of negation and disjunction: $!\varphi \equiv \neg\neg\varphi$ and $?\varphi\equiv \varphi \lor \neg \varphi$. So we do not have to add these operators to our language in order to make our language more expressive.

This concludes our introduction to \textsf{InqB}. In the next section we will discuss how we implemented \textsf{InqB} in Haskell. 


