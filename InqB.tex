\section{Inquisitive Semantics}\label{sec: InqB}
% Language
Any standard first-order language $\mathcal{L}$, consisting of a set of function symbols $\mathcal{F}_\mathcal{L}$ and a set of relation symbols $\mathcal{R}_\mathcal{L}$, is also a language of \textsf{InqB}. In our model checker we do not concern ourselves with function symbols, therefore we will not mention them in the remainder of this report. As for the constants in a language, we will assume that for each individual in the domain of a model we will have a constant in our language. We define models of \textsf{InqB} below.

% Model
\begin{defi}
%Let $\mathcal{L}$ be a standard first-order language that consists of a set of relation symbols $\mathcal{R}_\mathcal{L}$ and a set of function symbols $\mathcal{F}_\mathcal{L}$, each with an associated arity $n\geq 0$. We will refer to $0$-place functions as individual constants. 
An \textsf{InqB} model for a first-order language $\mathcal{L}$ is a triple $M=\langle W,D,I\rangle$, where:
\begin{itemize}
\setlength\itemsep{-0.3em}
    \item $W$ is a non-empty set of possible worlds;
    \item $D$ is a non-empty set of individuals;
    \item $I$ is a map that associates every $w\in W$ with a first order structure $I_w$ such that:
    \begin{itemize}
    \setlength\itemsep{-0.3em}
        \item for every $w\in W$, the domain of $I_w$ is $D$;
        \item for every $n$-ary relation symbol $R\in \mathcal{R}_{\mathcal{L}}$, $I_w(R)\subseteq D^n$;
    \end{itemize}
\end{itemize}
\end{defi}

% info states
% propositions
Before giving the semantics, we introduce some terminology. Instead of worlds, inquisitive semantics takes sets of worlds as primitive. A set of worlds is called an information state. A proposition, then, consists of a set of sets of worlds instead of a set of worlds as in classical logic.

\begin{defi}
 Let $M=\langle W,D,I\rangle$ be a model. An information state $s$ is a set of possible worlds $s\subseteq W$. A proposition $P$ is a non-empty, downwards closed set of information states.
\end{defi}

The information states in a proposition correspond to the states in which the issue raised by a proposition is resolved. As smaller information states provide \emph{more} information, the requirement that propositions are downwards closed make sense. 

In classical logic, logical operations correspond to certain algebraic operations. This is also the case for inquisitive semantics. We will characterize the semantics of \textsf{InqB}


% Algebraic foundations
% Semantics
% projection operators
