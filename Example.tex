In Figure \ref{fig:Example Model} we have given a visual representation of an \textit{InqB} model $M$. The grey areas correspond to the two alternatives, i.e., maximal elements, of the proposition corresponding to the formula $?!(Ra \lor Rb)$. Note that this formula is inquisitive as there are more than one alternatives. However, as the alternatives together cover the whole universe, we see that it is not informative.

\begin{figure}[h]
    \centering
    \begin{tikzpicture}[>=latex,scale=.6]
    
    % Possibilities    
    \draw[opaque,rounded corners]     
    (-1.8,0) -- (-1.8,1.8) -- (1.8, 1.8) -- (1.8,.2)
    -- (-.2, .2)
    -- (-.2,-1.8) -- (-1.8,-1.8) -- (-1.8,0);
    \draw[opaque,rounded corners] (.2,-.2) rectangle (1.7, -1.7);
    
    % Indices
    \draw (-1,1) node[index gray] (yy) {$1$};
    \draw (1,1) node[index gray] (yn) {$2$};
    \draw (-1,-1) node[index gray] (ny) {$3$};
    \draw (1,-1) node[index gray] (nn) {$4$};
  	
  	\end{tikzpicture}
    \caption{The model $M$ with the proposition $[?!(Ra \lor Rb)]$.}
    \label{fig:Example Model}
\end{figure}

\noindent In Section \ref{sec:Models} we have seen how we can represent this model in Haskell, i.e. $M$ corresponds to \verb|myModel|. Similarly, we have already seen the implementation of $?!(Ra \lor Rb)$ as \verb|myForm|. Lastly, we can compute the alternatives of the proposition $[?!(Ra \lor Rb)$ in GHCI as follows:
\begin{showCode}
*InqBSemantics> alt myModel myForm
\end{showCode}
\noindent This will return the following list of lists, which corresponds exactly to the alternatives in the figure.
\begin{showCode}
[[1,2,3],[4]]
\end{showCode}