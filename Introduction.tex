\section{Introduction}\label{sec: Introduction}
Inquisitive semantics is a relatively new framework in which information exchange can be analysed. In addition to declarative natural language sentences, questions can also be analysed in this semantic framework. In this report we describe how we build a model checker for the most basic version of inquisitive semantics, \textsf{InqB}.

We have two goals in mind. Firstly, we want to be able to evaluate formulas relative to specified models. And secondly, we want to check several theorems using QuickCheck.

In Section \ref{sec: InqB} we will give a concise introduction to \textsf{InqB}. Section \ref{sec:InqB in Haskell} concerns itself with the implementation of \textsf{InqB} and our model checker in Haskell. In Section \ref{sec:QuickCheck} we check several well-known facts of \textit{InqB}, after which we give our conclusion in Section \ref{sec:Conclusion}.

